\documentclass[a4paper, 12pt]{article}
\author{Victor Manuel Castillo Tamayo C111}
\title{Proyecto Moogle!}
\date{Julio, 2023}
\begin{document}
	\maketitle
	\section{Introduccion}\label{sec:intro}
		A modo de presentacion este documento esta dirigido a la explicacion del funcionamiento del codigo tras el programa, resumiendo el metodo utilizado para la busqueda del query y la obtencion de la porcion del documento donde se encuentra este. Para esto damos a conocer las distintas clases y metodos utilizados para el correcto funcionamiento de este.
	
	\section {Desarrollo}
		Este proyecto esta implementado de la siguiente forma:
\begin{itemize}
\item Carga de archivos:
Con la clase “DatosArchivos”, se procesa el contenido de la base de datos. Primero obtiene 
la ruta de los documentos y los nombres de estos, los lee, convierte el contenido a 
minúsculas y elimina los caracteres que dificultan la búsqueda, separando así cada palabra 
teniendo en cuenta los espacios en blanco.
También se encuentran datos como:
		\begin{itemize}
\item Cantidad de documentos en la base de datos
\item La frecuencia de cada palabra (cantidad de veces que se repite).
\item  Palabra con mayor frecuencia.
\item Cantidad de documentos donde aparece cada palabra
		\end{itemize}
\item IDF – TF:
Con la clase “CarpetaDeDatos”, pasamos al cálculo de IDF – TF de cada palabra. Los 
resultados serán guardados en diccionarios (IDFS – TFS), otorgando cada valor a la palabra 
en cuestión.
		\begin{itemize}
\item El valor IDF está dado por la fórmula:
	\begin{center}
		$$
			IDF=log_{10}{\frac{TD+1}{CD+1}}
		$$
	\end{center} 
Donde TD es el total de documentos existentes en la base de datos, y CD la cantidad de 
documentos donde aparece la palabra en cuestión.
\item El valor TF está dado por la fórmula:
	\begin{center}
		$$
			\frac{F}{maxF}
		$$
	\end{center}

Donde F es la frecuencia de la palabra en cuestión, y maxF la cantidad de documentos 
donde aparece.
		\end{itemize}
\item Motor:
El la clase “Motor” desarrollamos el modelo vectorial encontrando el “peso” de cada
palabra dado por la fórmula:
	\begin{center}
		$$
			\frac{{(IDF*TF)}^2}{{IDF}^2*{TF}^2}
		$$
	\end{center}
 Hallara y comparara estos datos antes obtenidos con la “consulta” insertada por el 
 Usuario. Devolviendo así un fragmento de cada texto con la consulta insertada.
\item Métodos utilizados:
En la clase “Útiles” se desarrollan varios métodos para la facilitación de los procesos 
anteriores. Estos son:
		\begin{itemize}
\item ConsultaSinOperadores:
Si la consulta tiene alguno de los operadores especificados, devuelve una lista de arrays
con la(s) palabra(s) que contiene el operador junto a este.
\item LimpiarTexto:
Este método elimina loa caracteres incomodos a la hora de leer los textos.
\item Encuentra:
Busca entre todas las palabras en los datos la que mayor índice de coincidencias tenga.
\item MaximoIndiceDeCoincidencias:
Método que devuelve el índice máximo de coincidencias entre dos palabras.
\item Direccion:
Obtiene la dirección desde donde se ejecuta la aplicación.
\item ArchivosEnCarpeta
Método que busca los archivos.
\item ExtraerPalabras:
Extrae las palabras de la consulta en minúscula.
\item ConsultaValida
Analiza y valida la consulta.
\item PalabrasSinRepetir:
Método para no repetir las palabras a procesar
		\end{itemize}
\end{itemize}
\end{document}